\documentclass[runningheads]{llncs}
\usepackage[spanish]{babel}
\usepackage[utf8]{inputenc}
\usepackage[T1]{fontenc}
\usepackage{graphicx}
\usepackage{amsmath}
\usepackage{amssymb}

\begin{document}

\title{PRISM: Verificación Formal de Sistemas con Modelos Probabilísticos}

\author{
    Angeles Carrara\and
    Julieta Paola Storino\and
    Mateo Carranza Velez}

\authorrunning{Carrara, Storino, Carranza-Velez}

\institute{
    Facultad de Matemática, Astronomía, Física y Computación.\\
    Universidad Nacional de Córdoba.
    \email{\{mateocvelez,julieta\_storino,angeles.carrara\}@mi.unc.edu.ar}}

\maketitle

\begin{abstract}
Este documento presenta una introducción a la herramienta PRISM\cite{KNP11}, un entorno para el modelado y análisis formal de sistemas con comportamientos aleatorios o probabilísticos, que permite la verificación rigurosa de propiedades cuantitativas y cualitativas sobre modelos como cadenas de Markov (discretas y continuas), procesos de decisión de Markov y autómatas probabilísticos. Se abordan los principales aspectos relacionados con su desarrollo y aplicación, incluyendo: el contexto de creación de la herramienta, sus objetivos fundamentales, una descripción de sus funcionalidades desde la perspectiva del usuario, los componentes técnicos que la sustentan, casos de estudio documentados —tanto exitosos como con limitaciones—, una comparación con otras herramientas similares en el ámbito de la verificación formal, y un análisis detallado de un caso de estudio seleccionado.
\end{abstract}

\section{Introducción}

Los sistemas informáticos se encuentran cada vez más integrados en múltiples áreas del conocimiento, enfrentando riesgos cada vez más críticos y de mayor impacto. Estos riesgos abarcan desde errores que pueden poner en peligro vidas humanas —como sobredosis de radiación provocado por condiciones de carrera en sistemas médicos \cite{LT93}—, hasta pérdidas económicas multimillonarias —como autodestrucción en cohetes espaciales provocado por conversión incorrecta de punto flotante a entero en sistemas de referencia inercial \cite{Lan96}—. Frente a esta realidad, se vuelve esencial garantizar sistemas más robustos, confiables y con una tolerancia al fallo mínima. Para ello, es necesario recurrir a modelos formales que permitan describir con precisión las partes relevantes del sistema, conservando un nivel de abstracción que facilite su análisis y verificación.

\section{Antecedentes}

Existen diversas técnicas para verificar la corrección funcional de un sistema, entre las que se incluyen la demostración de teoremas, la simulación, la prueba (testing), y la verificación de modelos (model checking). En nuestro informe, nos enfocamos en esta última, una técnica que consiste en explorar exhaustivamente todos los estados posibles del sistema mediante un enfoque de fuerza bruta, con el objetivo de comprobar que un modelo dado cumple efectivamente con una propiedad específica \cite{BK07}. Este enfoque tradicional de verificación de modelos busca proporcionar una garantía completamente precisa sobre el cumplimiento —o no— de una propiedad en el modelo analizado. Sin embargo, en muchos escenarios no es posible ni deseable ofrecer garantías absolutas. Por ejemplo, en protocolos de comunicación o redes inalámbricas, es necesario considerar una cierta probabilidad de pérdida de mensajes.

La verificación de modelos probabilísticos (Probabilistic Model Checking) es una extensión de la verificación de modelos clásica que nos permite establecer la validez de una propiedad dentro de un modelo dado. Es una técnica automática de verificación formal diseñada para analizar sistemas que presentan comportamientos estocásticos.
A diferencia de la verificación tradicional, cuyo resultado es estrictamente booleano (la propiedad se cumple o no), la verificación probabilística permite obtener valores cuantitativos, como la probabilidad de alcanzar un estado crítico o el tiempo esperado hasta que ocurra un evento. Esta capacidad resulta especialmente útil en el análisis de sistemas que interactúan con entornos inciertos, como redes de comunicación, sistemas embebidos y protocolos distribuidos.

Para comprobar el modelo de un sistema que muestra un comportamiento estocástico, primero necesitaremos construir un modelo probabilístico formal de ese sistema. Existen varias representaciones de modelos probabilísticos de uso común para sistemas estocásticos, pero nos vamos a enfocar en los principales modelos soportados por la herramienta a desarrollar.

\textbf{Cadenas de Markov de Tiempo Discreto (DTMC)}: Son modelos en los que el comportamiento del sistema evoluciona en pasos discretos, y en cada paso existe una distribución de probabilidad sobre los posibles estados siguientes. Se representan como grafos dirigidos etiquetados con probabilidades de transición. Para cada estado, las probabilidades de transición deben sumar $1$\cite{KNP02}. Estos modelos son apropiados para sistemas donde el tiempo se modela como secuencias de eventos discretos y completamente probabilísticos, como protocolos de comunicación simples o sistemas sin decisiones controlables.

\textbf{Procesos de Decisión de Markov (MDP)}: Extienden DTMC introduciendo no determinismo, lo que permite modelar elecciones o decisiones controlables dentro del sistema. En cada estado, el sistema puede tener varias acciones posibles, y cada acción induce una distribución de probabilidad sobre los estados sucesores. Este modelo es útil para representar la interacción entre el entorno (probabilístico) y un controlador (no determinista). Permiten modelar, por ejemplo, sistemas con múltiples componentes que operan en paralelo de manera asincrónica, donde ciertas decisiones dependen del contexto de ejecución.

\textbf{Cadenas de Markov de Tiempo Continuo (CTMC)}: Las transiciones entre estados ocurren después de un tiempo continuo aleatorio, no en pasos discretos. Cada transición está asociada con una tasa que determina la probabilidad de moverse de un estado a otro en un cierto tiempo, siguiendo una distribución exponencial. Este tipo de modelos es adecuado para analizar sistemas donde los eventos ocurren de forma continua en el tiempo, como redes de colas o sistemas biológicos.

Una vez que el sistema está representado por un modelo, queremos comprobar si este cumple una especificación formal. Para ello, se pueden utilizar diversos lenguajes de lógica temporal, pero nuevamente nos enfocaremos en los más relevantes y representativos en el contexto de la herramienta que se pretende construir.

\textbf{Lógica de Árbol Computacional Probabilística (PCTL)}: Es introducida por Hansson y Jonsson en \cite{HH94} como una extensión probabilística de la lógica temporal CTL. Su característica distintiva es el uso del operador probabilístico $P$, que generaliza los cuantificadores $\forall$ y $\exists$ de CTL para incorporar condiciones sobre la probabilidad de ocurrencia de ciertos eventos en un sistema estocástico. Por ejemplo, la fórmula
\[a\to P_{\sim p}[F^{\le s} b]\]
se interpreta como "si $a$ ocurre, entonces la probabilidad de que $b$ ocurra en los próximos $s$ pasos satisface la relación $\sim p$". La sintaxis de PCTL se divide en fórmulas de estado y fórmulas de camino, definidas de la siguiente manera:

\begin{itemize}
    \item \textit{Fórmulas de estado}: \( \phi ::= \text{true} \mid a \mid \phi \land \phi \mid \neg \phi \mid P_{\sim p}[\psi]\)
    \item \textit{Fórmulas de camino}: \(\psi ::= X\ \phi \mid \phi \ U^{[k_1,\ k_2]} \psi \mid \phi \ U \phi \mid F\ \phi \mid G\ \phi\)
\end{itemize}
Donde:
\begin{itemize}
    \item\(a\) es una proposición atómica,
    \item\(p \in [0,1]\) es un umbral de probabilidad,
    \item\(\sim \in \{<, >, \leq, \geq\}\) es un operador de comparación,
    \item\(k_1,k_2 \in \mathbb{N}\) indica una cota temporal (en pasos discretos).
    \item \(X=next\), \(F=future\), \(G=globally\), \(U=until\).
\end{itemize}

\textbf{Lógica Estocástica Continua (CSL)}: Es introducida por Aziz et al. en \cite{ASSB96} y luego extendida por Baier et al. en \cite{BKH99} como una extensión de la lógica anterior con la capacidad de contar con un período variable de transición entre un estado y otro. Sus operadores son similares a los de PCTL pero sustituyen los valores discretos de tiempo por valores continuos. CSL incorpora un operador de estado estacionario, denotado por $S_{\sim p}[\phi]$, que permite razonar sobre el comportamiento a largo plazo del sistema. Esta expresión indica que la probabilidad en estado estacionario de que la fórmula $\phi$ sea verdadera satisface la relación $\sim p$.

\section{Descripción de la herramienta del lado del usuario}

Prism es una herramienta que cuenta con interfaz gráfica, pero que también puede usarse sobre línea de comando. La interfaz gráfica está compuesta por cuatro pestañas cada una con una función diferente. Estas son 'Model', 'Properties', 'Simulator' y 'Log'.

En la pestaña 'Model' es donde se especifica el modelo, en el lenguaje PRISM. En la primera línea debemos indicar el modelo probabilístico que vamos a utilizar. Este puede ser 'dtmc', 'ctmc', 'mdp', 'pta', 'pomdp' o 'popta'. Un modelo (no confundir con los modelos nombrados recién) en Prism está formado por uno o más módulos. Cada módulo contiene variables, que describen el estado del módulo. Dentro de cada módulo, se describen sus transiciones, que dependen del estado actual del módulo, y posiblemente de otros módulos. En DTMC o MDP, cada transición describe la probabilidad de de pasar de un estado a otro. En CTMC, hay que especificar tasas en lugar de probabilidades. También cuenta con un sistema de costos y recompensas.
Para una descripción más detallada del lenguaje, leer la documentacón [TODO: poner ref]

En la pestaña 'Properties' es donde se especifican las propiedades que que queremos verificar. Estas deben escribirse en el lenguaje de especificación de propiedades PRISM, que soporta varias lógicas temporales probabilísticas, tales como PCTL, CSL, LTL probabilístico y PCTL*. Estas propiedades pueden ser de tipos. En las del primer tipo, le preguntamos al modelo si cierta propiedad es cierta o no. Por ejemplo, que la probabilidad de llegar a cierto estado es mayor o igual a 0.5. En las del segundo tipo, le pedimos al modelo que calcule algún valor. Por ejemplo, la probabilidad de llegar a cierto estado o tiempo esperado hasta que ocurra algo. Para calcular estos valores esperados se usan los costos y recompensas.
Para una descripción más detallada del lenguaje, leer la documentacón [TODO: poner ref]

En la pestaña 'Simulator' podemos realizar simulaciones del modelo. En estas simulaciones, podemos decirle a la herramienta que haga una cantidad específica de pasos aleatoriamente o podemos elegir nosotros manualmente las transiciones. Cuenta con una tabla que nos permite ver los valores de las variables en cada paso. También se pueden generar gráficos de estas simulaciones.

La herramienta permite visualizar los modelos, mediante de la generación de archivos .dot de los grafos asociados a los modelos.

PRISM permite también especifcar propiedades no probabilísticas, en CTL o LTL y es capaz de reportar contraejemplos cuando una propiedad no se cumple.

\section{Caso de estudio: Cruzando el río probabilístico}

Para ilustrar el funcionamiento de la herramienta, vamos a modelar una versión probabilística del problema de cruzar el río. [ref] Este es un problema clásico en la cultura popular. La formulación que proponemos es la siguiente:

Un granjero quiere cruzar al otro lado del río, junto con un lobo, una oveja y una lechuga. Para hacerlo, cuenta con un bote con el cual puede cruzarlo con a lo sumo un objeto más. En cada momento, si hay $e$ objetos de ese lado del río, elije qué hacer con igual probabilidad entre las $e+1$ opciones, que son cruzar el río solo o elegir uno de los $e$ objetos y cruzar con él. Si mientras está cruzando quedan del mismo lado el lobo y la oveja o la oveja y la lechuga, con probabilidad $p$ alguien resulta comido (la oveja en el primer caso, la lechuga en el segundo, y si ocurren ambos a la vez, cualquiera de ellos con igual probabilidad). Calcular la probabilidad de que el granjero pueda llegar al otro lado con los tres objetos.

Nuestro modelo de este problema en PRISM no fue incluído en el texto de este trabajo, pero puede encontrarse al pie de página. [poner] Para simplicar el modelado, usamos que en PRISM el no determinismo en una DTMC se traduce a una transición aleatoria con igual probabilidad entre todas las opciones posibles en ese momento. No incluiremos su grafo pues tiene $72$ estados y $124$ transiciones.

En las formulaciones usuales de este problema se suele usar $p=1$, es decir, siempre que alguien pueda comer, lo va a hacer. En esta versión, la probabilidad de poder cruzar es de aproximadamente $0.62\%$. En el otro extremo, si $p=0$ entonces nadie resulta comido y por lo tanto la probabilidad es del $100\%$. En la figura 1 podemos ver cómo varía la probabilidad de cruzar en función de $p$, para los valores $p=0.1k$ con $k$ entero entre $0$ y $10$. %Todo: incluir figuara

También usamos la herramienta para calcular cómo va incrementando la probabilidad de poder cruzar en función de la cantidad de pasos que realizamos. En la figura 2 pueden verse estas probabilidades para $p=\frac{1}{2}$ y con cantidades de pasos entre $0$ y $100$. Las mesetas que se forman debido a que solo cada $4$ pasos el granjero llega al lado derecho del río.

Además, verificamos las siguientes propiedades 'En algún momento se comen a la oveja con probabilidad $1$' y 'En algún momento se comen a la oveja'. Observar que la primera es probabilística y la segunda no. Al verificarlas, la primera resulta verdadera pero la segunda resulta falsa. Esto es porque existe un camino infinito en el que no se comen a la oveja, pero este tiene probabilidad $0$ de ocurrir. PRISM no reporta contraejemplos en estos casos infinitos. Si en cambio, verificamos que 'Es posible que crucen todos al otro lado' (que tampoco es probabilística), esto resulta verdadero y PRISM nos da el testigo de cómo hacerlo mendiante el simulador.

Usando el sistema de recompensas de PRISM, calculamos el valor esperado de la cantidad de cruces que ocurren hasta que se comen a alguien, y hasta que se coman a la oveja. Para $p=\frac{1}{2}$ estos valores dieron aproximadamente $6.14$ y $14.52$. La cantidad esperada de cruces que ocurren hasta que resolvemos el problema o es imposible de resolver (porque se comieron a alguien) es de $5.55$ cruces.

\begin{thebibliography}{8}

\bibitem{KNP11}
Kwiatkowska, M., Norman, G., Parker, D.: PRISM 4.0: Verification of Probabilistic Real-time Systems. In: Gopalakrishnan, G., Qadeer, S. (eds.) CAV 2011. LNCS, vol. 6806, pp. 585--591. Springer, Heidelberg (2011)

\bibitem{KNP02}
Kwiatkowska, M., Norman, G., Parker, D.: PRISM: Probabilistic Symbolic Model Checker. In: Field, T., Harrison, P.G., Bradley, J.T., Harder, U. (eds.) TOOLS 2002. LNCS, vol. 2324, pp. 200--204. Springer, Heidelberg (2002)

\bibitem{BK07}
Baier, C., Katoen, J.-P.: Principles of Model Checking. MIT Press, Cambridge (2007)

\bibitem{LT93}
Leveson, N.G., Turner, C.S.: An Investigation of the Therac-25 Accidents. IEEE Computer \textbf{26}(7), 18--41 (1993)

\bibitem{Lan96}
Le Lann, G.: The Ariane 5 Flight 501 Failure -- A Case Study in System Engineering for Computing Systems. Rapport de recherche \textbf{3079} (1996)

\bibitem{HH94}
Hansson, H., Jonsson, B.: A Logic for Reasoning about Time and Probability. Formal Aspects of Computing \textbf{6}(5), 512--535 (1994)

\bibitem{ASSB96}
Aziz, A., Sanwal, K., Singhal, V., Brayton, R.: Verifying Continuous Time Markov Chains. In: CAV 1996. LNCS, vol. 1102, pp. 269--276. Springer, Heidelberg (1996)

\bibitem{BKH99}
Baier, C., Katoen, J.-P., Hermanns, H.: Approximate Symbolic Model Checking of Continuous-Time Markov Chains. In: CONCUR 1999. LNCS, vol. 1664, pp. 146--161. Springer, Heidelberg (1999)

\end{thebibliography}
\end{document}